\addchap{Lista de Símbolos e Abreviações}
\begin{refsection}

\begin{itemize}
\item $X_{i}$: valor da i-ésima observação da variável em estudo;
\item $\mu$: média aritmética de uma população de dados.
\item $\sigma$: desvio-padrão em uma população de dados;
\item $\alpha$: nível de significância, probabilidade do erro tipo I;
\item $k$: número de classes na distribuição de frequência;
\item $\Omega$: espaço amostral;
\item $\varepsilon$: resíduo ou erro amostral;
\item $\Sigma$: somatório de uma amostra de dados;
\item $H_{0}$: hipótese nula ou nulidade;
\item $H_{A}$: hipótese alternativa;
\item $N$: tamanho da população; 
\item $n$: tamanho da amostra;
\item $N_{h_{i}}$: tamanho da população no h-ésimo estrato;
\item $n_{h}$: tamanho da amostra no h-ésimo estrato;
\item $t$: estatística do teste t student;
\item $P-Valor$: nível crítico amostral ou descritivo, significância estatística de um teste;
\item $\bar{X}$: média aritmética de uma amostra de dados;
\item $\widetilde{X}$: mediana de uma amostra de dados; 
\item $\bar{X}_{h}$: média harmônica de uma amostra de dados;  
\item $\bar{X}_{g}$: média geométrica de uma amostra de dados;
\item $CV$: coeficiente de variação de uma amostra de dados;
\item $SPSS$: Statistical Package for the Social Sciences;
\item $SAS$: Statistical Analysis System;
\item $RStudio$: é um software livre de ambiente de desenvolvimento integrado para R;
\item $ABNT$: A Associação Brasileira de Normas Técnicas;
\item $IBGE$: Instituto Brasileiro de Geografia e Estatística;
\item $DETRAN-PA$: Departamento de Trânsito do Estado do Pará;
\item $DATASUS$: Departamento de informática do Sistema Único de Saúde do Brasil; 
\item $FAPESPA$: Fundação Amazônia de Amparo a Estudo e Pesquisas;
\item $RAIS$: Relação Anual de Informações Sociais;
\item $FINBRA$:  base de dados formado pelas informações das declarações recebidas pelo Tesouro Nacional; 
 \item $V.A.D$: variável aleatória discreta;
\item $V.A.C$: variável aleatória contínua;
\end{itemize}




\printbibliography[heading=subbibliography]
\end{refsection}