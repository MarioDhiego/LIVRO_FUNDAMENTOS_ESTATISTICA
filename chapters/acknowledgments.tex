\addchap{Agradecimentos}
\begin{refsection}


Gostaríamos de expressar nossos mais sinceros agradecimentos a todas as pessoas que contribuíram para a produção deste livro., em especial ao Excelentíssimo Governador \textbf{Helder Zaluth Barbalho} por ter um visão estratétiga clara para o desenvolvimento do Estado do Pará e o compromisso com a sustentabilidade, atuando incansavelmente para o cresscimento econômico e social da região amazônica sem desconsiderara a preservação do meio ambiente e a valorização da Ciência, reconhecendo que a Estatística tem papel fundamental neste desenvolvimento siustentável almejado.\vst

Expressamos nossa gratidão à Secretaria de Segurança Pública do Estado do Pará que apoiou e, parcialmente, financiou este projeto. O compromisso com a sociedade e o desenvolvimento do setor governamental do Estado do Pará é inestimável. Esperamos que este compêndio seja um recurso valioso para todos os servidores públicos que buscam se capacitar para a tomada de decisões embasadas em dados, promovendo a governança eficiente e sustentável tão almejada em nossa Região Amazônica.\vst

Destacamos ainda nosso agradecimento à Autarquia de Trânsito do Estado do Pará (DETRAN-PA), que sob a chancela da sua Diretora geral, Dra. \textbf{Renata Mirella Freitas Guimaraes de Sousa Coelho}, generosamente permitiu o uso de suas dependências para importantes discussões e parte da produção deste livro.
\vst

%Ao Estatístico e servidor público, \textbf{Alexandre Rodrigues Ramos}, agradecimentos pelo gentil compartilhamento de seus conhecimentos técnicos que permitiram a revisão minuciosa deste livro. Suas sugestões e correções foram essenciais para garantir a qualidade e o bom entendimento do material apresentado. Sem a dedicação e o conhecimento compartilhados por esses indivíduos, essa obra não teria sido possível. Seus esforços e experiências enriquecem cada página deste livro, proporcionado ao leitores uma compreenção aprofundada das ferramentas estatísticas necessárias para a toamda de decisões informadas no setor público.

%\newpage

Este Livro é dedicado a todas as pessoas que, igual a nós, são apaixonadas pela Estatística, Matemática e Linguagem de Programação R, e me incentivaram a levar o conhecimento para as gerações iniciantes.
\vst

%Agradecemos as críticas e sugestões.


%\section*{Mário Diego Rocha Valente}




\printbibliography[heading=subbibliography]
\end{refsection}






\addchap{Epígrafe}

\begin{epigrafe}
\vspace{3cm}


\begin{minipage}{10cm}

\textit{``Se a Matemática é o pincel com que Deus desenhou o universo, então, a Estatística é a ferramenta humana criada para tentar entendê-lo.'' (\textbf{Waldenei Travassos Queiroz})}

\end{minipage}



\end{epigrafe}




