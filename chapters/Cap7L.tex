\chapter{Referências Bibliográficas}

BOISSEL, J. P. Planning of clinical trials. J Intern Med 2004; 255: 427-38.\vskip0.3cm

BROMAN, Karl, and Kara Woo. 2017. “Data Organization in Spreadsheets.” The American Statistician 72 (1): 2–10. https://doi.org/10.1080/00031305.2017.1375989.\vskip0.3cm

CRONBACH, L. J. Coefficient alpha and the internal structure of test. Psychometrika, v. 16, n. 3, p. 297-334, 1951.\vskip0.3cm

Dantas, C. A. B. Probabilidade: Um Curso Introdutório. Ed.
USP, São Paulo, 1997.

FELLER, W. Introdução a Teoria das Probabilidades e suas
Aplicações. Parte 1°: Espaços Amostrais Discretos, Edgard Blucher. São Paulo, 1976.\vskip0.3cm

FEIJOO, A. M. L. C. A pesquisa e a estatística na psicologia e na educação. Rio de Janeiro:
Centro Edelstein de Pesquisas Sociais, 2010, 109p. ISBN: 978-85-7982-048-9.\vskip0.3cm

GAL, I. Statistical tools and statistical literacy: the case of the average. Teaching
Statistics, 17(3), 97-99, 1995.\vskip0.3cm

HIEBERT, J. e LEFEVRE, P. Conceptual and procedural knowledge in mathematics: an
introductory analysis. In J. Hiebert (Ed.), Conceptual and procedural knowledge: the
case of mathematics, New Jersey, NJ: Lawrence Erlbaum, pp. 1-27, 1986.\vskip0.3cm

HOEL, P. G., PORT, S.C., STONE, C.S. Introdução a Teoria da Probabilidade. Rio de
Janeiro: Luter-Ciência, 1971.\vskip0.3cm

Hulley, Stephen B.; Cummings, Steven R.; Browner, Warren S. et al. Delineando a pesquisa clínica: uma abordagem epidemiológica. 2ª Ed. Porto Alegre: Artmed, 2003. p: 21-34.\vskip0.3cm

JAMES, B. R. . Probabilidade: Um curso em nível intermediário.
Projeto Euclides, IMPA, Rio de Janeiro, 1981. \vskip0.3cm


KENDALL, J. M. Designing a research project: randomised controlled trials and their principles. Emerg Med J. 2003 Mar;20(2):164-8. doi: 10.1136/emj.20.2.164. PMID: 12642531; PMCID: PMC1726034.\vskip0.3cm

MARCONI, Marina de Andrade e Lakatos, Eva Maria. Fundamentos de metodologia científica. 6ª ed. São Paulo, SP: Atlas, 2005.\vskip0.3cm

MEMÓRIA, J. M. P. Breve história da estatística, Embrapa Informação Tecnológica, Brasília, 2004, 111p. Texto para discussão, ISSN 1677-5473;21. \vskip0.3cm

MEYER, P. L. Probabilidade: Aplicações e Estatística. 2 ed.
Rio de Janeiro: Livros Técnicos e Científicos. Editora S.A. ,
1984.\vskip0.3cm


PADOVANI, Carlos Roberto. Bioestatística. São Paulo, Cultura Acadêmica:
Universidade Estadual Paulista, Pró-Reitoria de Graduação, 2012.\vskip0.3cm

RAO, C. R. Linear statistical inference and its applications. New York: Wiley, 1973.\vskip0.3cm

RAO, C. R. Statistics: A technology for the millenium. International Journal of Mathematical and Statistics Sciences, Brasília, v. 8, n. 1, p. 5-25, 1999.\vskip0.3cm

TUKEY, J. W. Exploratory data analysis. Reading: Addison–Wesley, 1977.\vskip0.3cm

SNEDECOR, G. W. Statistical methods applied to experiments in agriculture and biology. Ames: The Iowa State College Press, 1937. \vskip0.3cm

STIGLER, S. M. The history of statistics: the measurement of uncertainty before 1900. Cambridge, MA: Harvard University Press, 1986. \vskip0.3cm

STUDENT. The probable error of a mean. Biometrika, London, v. 6, p. 1-25, 1908a.\vskip0.3cm

STUDENT. Probable error of a correlation coefficient. Biometrika, London, v. 6, p. 302-310, 1908b.\vskip0.3cm

STEINS, C. E; LOESCH, C. Estatística Descritiva e Teoria das Probabilidades, 2011 série Didática, EDFURB.\vskip0.3cm

SILVA, Edna Lúcia da. e Menezes, Estera M. Metodologia da pesquisa e elaboração de dissertação. 3ª ed. Florianópolis: Laboratório de Ensino a Distância da UFSC, 2001.\vskip0.3cm

SILVA, Cassandra Ribeiro de O. Metodologia e organização do projeto de pesquisa: guia prático. Fortaleza, CE: Editora da UFC, 2004.\vskip0.3cm

ROSS, S. A. First Course in Probability. 3 ed. New York:
McMillan Publishing Company, 1988.\vskip0.3cm


WHIPPLE, George Chandler. Vital Statistics: An Introduction to the Science of Demography, 1923. 







