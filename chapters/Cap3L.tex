\chapter{Principais Métodos de Amostragem}

\section{Introdução}

O objetivo de todo projeto de pesquisa é, a partir do estudo de uma amostra, fazer inferência para uma determinada população. Logo, para que i inferência estatística seja válida, é necessário que a amostra que a amostra seja representativa da população de onde foi retirada, de maneira que os resultados encontrados sejam os mais fidedignos possíveis. (CALLEGARI-JAQUES , 2003). Assim, é necessário que o autor do projeto atente para os fatores importantes, tais como o método de amostragem e o cálculo do amostral, pois, amostras mal selecioandas e de tamanho inadequado, compormetem o resultado da pesquisa, uma vez que não representam fielmente a população (DAWSON, 2003).
\vskip0.3cm

Nesse sentido, é importante realçar que o cálculo do tamanho amostral
tem sido um dos maiores desafios para aqueles que desejam conduzir um experimento científico, pois nem sempre os métodos para esse cálculo se apresentam de maneira simples e compreensíveis, o que traz ansiedade e dúvidas ao pesquisador (BEIGUELMAN, 2002).
\vskip0.3cm

Para Silva (2004) os levantamentos por amostragem, tem com a
finalidade de produzir instantaneos das realidades estudadas, os
levantamentos por amostragem reunem as seguintes caracteristicas
operacionais:

\begin{enumerate}
    \item[{i})] Aplicam-se conjuntos reais e finitos, compostos de
    elementos denominados de população de estudo.
    \item[{ii})] os elementos podem ser seres humanos, animais, árvores,
    fichas, prontuários, domicílios, áreas ou objetos.
    \item[{iii})] As caraterísticas ou atributos são observados em cada
    elemento e posteriormente agregados por meio de medidas
    estatísticas chamadas parâmetros ou valores populacionais.
\vskip0.3cm
    \item[{iv})] Os dados são coletados em amostras das populacões de
    estudo, e as medidas calculadas (estimativas) passam a ser a
    informação disponível para os valores populacionais
    desconhecidos.
\end{enumerate}


De acordo com Silva (2004), os levantamentos pode ter finalidade
\textbf{descritiva}, limitando-se a estimar frequências de
elementos com determinada propriedade ou estimar médias e
variâncias de características quantitativas.\vskip0.3cm

Outros levantamentos, denominados \textbf{analíticos ou
investigaões}, definem grupos de comparação e, além de estimar,
procuram detectar relações entre as características, buscando
aumentar as explicações para o objeto pesquisado (SILVA,
2004).\vskip0.3cm

\inic Existem alguns tipos de levantamentos, técnicas ou métodos
recomendados para a extração de uma amostra, a fim de que esta
possa ser considerada suficientemente representativa da população.
Essas técnicas chamadas de amostragem podem ser classificadas em
duas categorias: as Probabilísticas e Não
Probabilísticas.\vskip0.3cm




A amostragem é dita \textbf{Probabilística ou Casual} quando se
conhece a probabilidade (diferente de zero) que cada elemento da
população tem em fazer parte da amostra. Portanto, para obtenção
de uma amostragem probabilística é necessário que a população seja
finita e que todos os elementos sejam acessíveis. As principais
técnicas de amostragem probabilísticas são: \textbf{amostragem
casual ou aleatória simples}, \textbf{amostragem sistemática},
\textbf{amostragem estratificada} e \textbf{amostragem por
conglomerados} (MATTAR, 2001).\vskip0.3cm

A amostragem é dita \textbf{Não-Probabilística} ou \textbf{Não
Casual} quando nem todos os elementos da população têm
probabilidade conhecida, e diferente de zero, de pertencer à
amostra. Apesar das vantagens do uso da amostragem probabilística,
existem situações onde sua utilização é praticamente impossível.
Isso ocorre principalmente nas situações onde não se tem acesso a
toda população e nos casos em que a população é formada por
material contínuo, tornando impraticável um sorteio rigoroso.
Existem basicamente três tipos de amostragem não-probabilística:
\textbf{amostragem por conveniência}, \textbf{amostragem
intencional} e \textbf{amostragem por quotas} (MATTAR,
2001).\vskip0.3cm


De modo geral as pesquisas utilizam amostras obtidas através de regras probabilísticas. A utilização incorreta da estatística pela amostragem (algumas vezes devido à ingenuidade, porem muitas vezes proposital), mediante a não observância destas regras, pode conduzir a resultados tendenciosos.\vskip0.3cm

\newpage 

Muitas pesquisas partem de amostragem para obter resultados que possam explicar determinados fenômenos. A fim de que uma amostra seja efetivamente representativa da população devem-se adotar técnicas recomendadas para selecionar os elementos que irão compor a amostra, para que não se crie alguma tendenciosidade (o mesmo que viés), que vem a ser uma distorção sistemática entre a medida de uma variável estatística e o valor real da grandeza a estimar, devida a imperfeição ou deformação da amostra que serve de base para a estimativa.\vskip0.3cm


O uso de amostras, substituindo toda a população, em um trabalho de análise estatística se justifica sempre que:
\begin{itemize}
\item Não se tem acesso a todos os elementos da população;
\item O custo de trabalhar com toda a população for muito alto;
\item O tempo para a coleta de dados for um fator relevante;
\item A amostra é a única opção no caso que o estudo exige a destruição ou contaminação dos elementos pesquisados;
\item Quando a população for infinita.
\end{itemize}


O uso de amostras pode proporcionar maior rapidez no trabalho, porém exige determinados cuidados na obtenção das mesmas. Uma amostra de boa qualidade deve ser:

\begin{itemize}
\item Precisa: exatidão dos resultados das medições obtidas na mostra;
\item Eficiente: quando produz resultados de maior precisão ou mesma precisão quando comparado a outro método, porém de menor custo;
\item Correta: ausência de vício ou sem erros sistemáticos.
\end{itemize}



\newpage
\subsection{Fatores que Afetam o Tamanho da Amostra} 
\inic Existem alguns fatores que afetam o tamanho da amostra em projetos de pesquisas científicas nas diversas áreas.

\begin{itemize}
\item \textbf{Objetivo da Amostra}: estudos descritivos costumam exigir amostra com menror número de particpantes;
\item \textbf{Tipo de variável}: as variáveis qualitativas exigem amostras maiores que as variáveis quantitativas, que exigem amostras maiores quanto maior for a variação nos dados amostrais; 
\item \textbf{Delineamento do Estudo}: estudo pareado requer uma amostra com metade do número de sujeitos, quando comparados aos estudos não-pareados;
\item \textbf{Valor Estimado para o Erro Alfa}: corresponde ao erro máximo que o pesquisador aceita cometer ao aplicar o teste estatístico para aceitar ou rejeitar hipótese nula. È o erro máximo que ele aceita para um erro falso-positivo. Na área das ciências da saúde é estipulado em 5\%. Quanto menor o erro alfa estipulado pelo pesquisador, maior será o tamanho estimado para a amostra. 
\item \textbf{Poder do Teste Estatístico}: corresponde a probabilidade de que o estudo detecte uma diferença real entre os grupos estudados. Traduz a probabilidade de o pesquisador cometer um erro falso-positivo. Na área das ciencias sa saude e arbitrado em 80\%, 85\% ou 90\%, que corresponde a um erro beta de 20\%, 15\% e 10\%.Quanto maior o tamanho da amostra, maior será o poder do estudo em detectar uma diferença ou um efeito real;
\item \textbf{O Tamanho da Diferença}: corresponde ao tamanho da verdadeira diferença que se dejesa discriminar como significativa, entre as médias da variável considerada no estudo. pequenas diferenças exigem amostras maiores;
\item \textbf{O Tamanho da População}: para pequenas populações o tamanho da amostra é diretamente proporcional ao tamanho da população. Para grandes populações, o tamanho da amostra não é influenciadao pelo tamanho da população, pois a mesma deverá ser considerada como ilimitada;
\item\textbf{ Dos Recursos e do Tempo Disponível}: é outro fator limitante que, não menos importante, pode influenciar no tamanho da amostra.
\end{itemize}


\subsection{Procedimentos Amostrais Iniciais}

\inic O cálculo do tamanho de uma amostra, impõe ao pesquisador
que ele especifique um valor predefinido para o erro amostral
(margem de erro), o qual deve ser pensado em termos de
probabilidades, pois, mesmo que uma amostra seja sificientemente
grande, ele não garante que suas características sejam exatamente
iguais a da população de onde foi retirada, uma vez que sempre
existe a probabilidade da randomização gerar uma amostra bem
diferente da população. Ou seja, a margem de erro exprime o valor
de quanto o pesquisador admite errar na avaliação dos parametros
estudados (FONTELLES et al, 2009).\vskip0.3cm

Inicia-se com a análise da situação em que não se pode determinar
o tamanho da população (N). Nesse caso, o tamanho mínimo da
amostra aleatória simples pode ser determinado através do cálculo
de $n_{0}$, considerado uma primeira aproximação para o cálculo do
tamanho da amostra, dado por


\begin{equation}\label{nzero}
    n_{o}= \left[  \frac{1}{\left(\epsilon_{0}\right)^2} \right]
\end{equation}

Sendo $\epsilon_{0}$ o erro amostral máximo tolerável.\vskip0.3cm

A expressão acima apresentada mantém fixo o nível de confiança de 95\% e a variância populacional no caso de maior heterogeneidade da população, ou seja, quando a proporção do evento na população em estudo é de 0,5. A fixação da proporção populacional do evento em 0,5, deve-se ao fato de ser esta a pior situação possível em termos de variabilidade populacional. \vskip0.3cm

Assim, pode-se considerar que a expressão (2.1) destina-se a três
situações: uma primeira, na qual não se conhece uma estimativa da
proporção do evento na população em estudo, uma vez que qualquer
que seja o valor da proporção, este dá origem a uma variabilidade
menor que aquela vinculada à proporção 0,5. Observa-se que, neste
caso é preciso maior cuidado com a determinação da amostra e,
conseqüentemente, a quantidade de elementos que a
comporão.\vskip0.3cm

Uma segunda situação na qual o valor de uma estimativa preliminar para a proporção do evento estudado é igual a 0,5 e, uma última, na qual o estudo destina-se à estimação da proporção de vários eventos da população, com pelo menos um dos eventos sem presença de uma estimativa anterior de sua proporção na população.\vskip0.3cm


\newpage
A seguir, apresenta-se uma tabela com aplicações da fórmula acima para alguns valores de erro amostral tolerável, a fim de se exemplificar a relação entre $\epsilon_{0}$ e uma primeira estimativa para o tamanho da amostra ($n_{0}$).




\begin{table}[!htb]
    \centering
    {
    \caption{Exemplos de tamanho de amostra ($n_{0}$) em função do erro amostral tolerável, variando de 1 a 5\%}
    \label{amostras}
    \vspace{0.1cm}
\begin{tabular}{c|c}
  \hline\hline
  % after \\: \hline or \cline{col1-col2} \cline{col3-col4} ...
  Erro Amostral $(\varepsilon_{(0)})$   & Tamanho Amostral ($n_{(0)})$ \\
  \hline\hline
  0,010    &  10.000 \\
  0,015    &  4.444 \\
  0,020    &  2.500 \\
  0,025    &  1.600 \\
  0,030    &  1.111 \\
  0,035    &  816 \\
  0,040    &  625 \\
  0,045    &  494 \\
  0,050    &  400 \\
  \hline\hline
\end{tabular}}
\\
\hspace{-1.0cm}
\end{table}

%\newpage

Conforme pode-se observar na tabela 2.1, quanto menor o erro
amostral tolerado pelo pesquisador, maior o tamanho da amostra
necessário para se atendê-lo. Considerando que o erro amostral
tolerável representa o quanto o pesquisador admite errar na
estimação dos parâmetros de interesse, ou seja, especifica o
intervalo em torno do valor que a estatística acusa, dentro do
qual encontra-se o verdadeiro valor do paramêtro que se deseja
estimar, quanto menor o erro amostral tolerado pelo pesquisador,
maior será o tamanho da amostra para que se possa obter essa maior
precisão da estatística.\vskip0.3cm



Assim, por exemplo, se o pesquisador tolerar no máximo um erro de 2\%, i.e., que o verdadeiro valor do parâmetro seja no máximo 2\% menor ou 2\% maior que o valor que a estatística acusa na amostra, ele terá que trabalhar com uma amostra aleatória composta por 2.500 indivíduos da população, ao passo que, se o pesquisador tolerar um erro amostral de 2,5\%, ele terá que trabalhar com uma amostra aleatória composta por 1.600 indivíduos da população e o verdadeiro valor do parâmetro da população estará no intervalo entre 2,5\% a menos até 2,5\% a mais do valor que a estatística acusa na amostra, com 95\% de probabilidde. Portanto, quanto maior a precisão que se deseja associar à estimativa estatística, maior o tamanho amostral necessário para atendê-la.\vskip0.3cm


\newpage
Conhecendo-se o tamanho da população N, pode-se corrigir o cálculo de $n_{0}$, obtido por \ref{nzero}, para se ter o tamanho da amostra aleatória simples, n, através da expressão:


\begin{equation}\label{nzero2}
    n= \left[ \frac{\left(N \times n_{0}\right)}{\left(N + n_{0}\right)} \right]
\end{equation}


Para exemplificar, Rocha (2013) fez um levantamento junto a
Delegacia de Crimes contra a Mulher de Macapá (AP) no  período de
agosto de 2011 a junho de 2012, do universo das ocorrências
registradas, há um total de 1.735 de ocorrências consideradas por
aquele órgão estatal como solucionadas. Desse total, 994 são
relativos à violência conjugal distribuídos da seguinte forma: 546
solucionadas, de agosto a dezembro de 2011 e 448 solucionadas, de
janeiro a junho de 2012. Para a realização do estudo optou-se em
retirar uma amostra representativa do total referente à violência
conjugal, por meio das fórmulas demonstradas anteriormente.

\vskip0.3cm


O valor do $\varepsilon_{0}=erro$ a ser adotado será de 5\%.
Portanto, substituindo o valor do erro na expressao 2.1

\begin{equation}\label{nzero}
    n_{o}= \left[ \frac{1}{\left(\epsilon_{0}\right)^2} \right] = \left[ \frac{1}{\left(0,05
    \right)^2} \right]= 400
\end{equation}

Em seguida, substituindo o resultado na formula 2.2,

\begin{equation}\label{nzero2}
    n=\left[\frac{N \times n_{0}}{N + n_{0}}\right]=
    \left[\frac{994 \times 400}{994 + 400}\right]=285
\end{equation}

Assim, faz-se necessário, apenas uma amostra de 285 ocorrências
registradas sobre violência conjungal, sendo esta,
estatísticamente representativa da população de 994.



\newpage
\section{Amostragem Probabilística}
\inic  Os Principais métodos de amostragem probabilísticas são: Aleatória Simples, Sistemática, Estratificada e Conglomerado.

\subsection{Amostragem Aleatória Simples (AAS)}

\inic Este tipo de amostragem também chamada de Simples ao Acaso,
Casual Simples, Elementar ou Randômica. Basei-se num processo
estritamente aleatório, na qual as unidades amostrais são
selecionadas com igual probabilidade $(\frac{1}{N})$, em que N é o
número total de unidades que compõem o espaço amostral, ou seja, a
população amostrada (QUEIROZ, 2012).\vskip0.3cm


A amostragem aleatória simples pode ser feita com reposição (caso em
que cada elementos da população pode entrar mais do que uma vez na
amostra) ou sem reposição (caso em que cada elemento da população
são pode entrar uma vez na amostra).\vskip0.3cm



Na amostra simples ao acaso sem reposição, as unidades são selecionadas independentemente entre si, e o número de amostras possíveis (nap) de tamanho igual a n é dada pela relação:


\begin{equation}\label{nap}
    nap=  \mathcal{C}_{N}^{n}= \left[\frac{N!}{n!(N-n)!} \right] = \left[\frac{N}{n}\right]^{-1}
\end{equation}

A amostra simples é um processo congruente, ou seja, quando n=N, as estimativas coincidem com os valores populacionais, recomendada para pequenas àreas com caracteristicas homogêneas com respeito as variáveis de interesse e com fácil estrutura de acessibilidade.

\subsection{Amostragem Estratificada (AE)}

\inic Muitas vezes a população se divide, em sub-populações,
subconjuntos ou estratos, sendo razoável supor que em cada estrato
a variável de interesse (sendo estudada) apresente um
comportamento substancialmente diverso. Por outro lado, pode-se
supor que o comportamento é razoavelmente homogêneo dentro de cada
estrato. Em tais casos, se o sorteio dos elementos da amostra for
realizado sem se levar em consideração a existência dos estratos,
pode acontecer que os diversos estratos não sejam convenientemente
representados na amostra, o que influenciara o resultado pelas
características dos estratos mais favorecidos pelo sorteio.\vskip0.3cm

Evidentemente, a tendência à ocorrência desta influência será
tanto maior quanto menor for o tamanho da amostra. Para evitar
este efeito, pode-se adotar uma amostragem estratificada.\vskip0.3cm

A amostragem estratificada consiste essencialmente em
pré-determinar quantos elementos da amostra serão retirados de
cada estrato. A pré-determinação pode ser feita de várias formas,
sendo as mais comuns chamadas de uniforme (onde se sorteia um
número igual de elementos em cada estrato) e proporcional (onde o
número de elementos sorteados em cada estrato é proporcional ao
número de elementos no estrato).\vskip0.3cm



A amostragem estratificada uniforme será recomendável se os
estratos da população forem pelo menos aproximadamente do mesmo
tamanho. Caso contrario, será preferível a estratificação
proporcional pelo fato de fornecer uma amostra mais representativa
da população.\vskip0.3cm

%\newpage

A estratificação pode levar em conta mais de um critério: por
exemplo, além do estado civil, poderiamos pré-determinar a
estratificação da amostra levando em conta faixas etárias (já que
dispomos de informação detalhada da distribuição dos indivíduos
por faixa etária nos censos de população)\vskip0.3cm

É importante observar, entretanto, que a precisão de uma amostra
não depende de unicamente da dimensão da população, mas também da
respectiva variabilidade. A variabilidade de um estrato é elevada,
quando os seus elementos têm características muito heterogêneas.
Tal situação implica que um estrato com maior variância deverá
levar à seleção de um maior número de unidades na amostra, quando
comparado com um estrato com a mesma dimensão populacional mas
menor variância (maior homogeneidade).\vskip0.3cm

Em resumo, quanto maior for o estrato, maior deve ser a amostra
respectiva. Mas se a variabilidade dentro de um estrato for maior,
maior deverá ser a respectiva sub-amostra. Este método otimiza a
amostra aplicada a um universo estratificado, razão pela qual
também é conhecida como distribuição estratificada
otimizada.\vskip0.3cm



\newpage
A amostragem estratificada consiste em dividir uma população de
tamanho N e L subpopulações constituídas de
$N_{1},N_{2},\ldots,N_{L}$ unidades, tal que não haja superposição
e, juntas, totalizem a população de tamanho N, ou seja:

\begin{equation}\label{N}
    N=\sum_{h=1}^{L}N_{h}, \ tal \ que \ h=1,2,\ldots,L
\end{equation}


As subpopulações, denominadas estratos, devem ter os valores
$N_{h}$ conhecidos, pois dentro de cada estrato, separadamente,
seleciona-se uma amostra de tamanho $n_{h}$. A grandeza amostral
para a população será igual a $n=n_{1}+n_{2}+\ldots+n_{L}$.
\vskip0.3cm

Assim, para se calcular o tamanho de amostra estratificada
proprocional a cada estrato da população, utiliza-se a seguinte
fórmula (QUEIROZ, 2012),

\begin{equation}\label{amostestra}
n_{h_{i}} = n \times \left ( \frac{N_{h_{i}}}{N} \right )
\end{equation}

onde, 

\begin{itemize}
\item $N$ é o número total de unidades em que a população foi dividida;
\item $N_{h}$ é o número de unidades em que o h-ésimo estrato foi dividido;
\item $n$ é o número de unidades de amostra a serem medidas considerando todos os estratos;
\item $n_{h}$ é o número de unidades de amostra a serem medidas no h-ésimo estrato;
\item $W_{h}={\frac{N_{h}}{N}}$ é o peso do h-ésimo estrato;
\item $\frac{n_{h}}{N_{h}}$= fator de amostragem no h-ésimo estrato; 
\item $L$ é o número de estratos;
\end{itemize}

Em relação à aplicação da amostragem, uma população pode ser classificada como finita ou infinita. Determinadas populações finitas podem ser consideradas infinitas desde que o tamanho da amostra seja no máximo 5\% da áream ou seja:

\begin{itemize}
    \item 1) Se $\frac{(N_{h}-n_{h})}{N_{h}} \geqslant 0,95$, a população é definida como Infinita; 
    \item 2) Para $\frac{((N_{h}-n_{h}))}{N_{h}} < 0,95$, a população é definida como Finita;
    \item O termo $\frac{((N_{h}-n_{h}))}{N_{h}}$ é denominado correção para populaão Finita.
\end{itemize}













%\newpage

A amostragem estratificada é recomendadável para o estudo de
populações que apresentem heterogeneidade entre as subpopulações
com referência à variável de interesse. Neste caso, a
estratificação pode propiciar um aumento no grau de precisão,
pois, torna possivel subdividir uma população heterogênea em
estratos que, individualmente, sejam homogêneos, resultando ganho
em eficiência e diminuição dos custos.\vskip0.3cm


A determinação dos estratos é feita em função das características
peculiares da população, onde, em muitos casos, os estratos já
estão fisicamente definidos. Por outro lado, existem casos em que
a delimitação dos estratos só é possível através de levantamentos
específicos.\vskip0.3cm


O processo para obter os estratos denomina-se estratificação e a
amostra é dita estratificada. Denomina-se amostra acidental
estratificada quando são selecionadas amostras inteiramente ao
acaso nos estratos.\vskip0.3cm

Conforme Cochran (1965), existem diversos critérios para se
utilizar o método de amostragem estratificada. Os principais são
os seguintes:


\begin{enumerate}
  \item[{1)}] A estratificação ideal é aquela realizada segundo a magnitude do valor da variável de interesse,
  obtendo-se gnho de precisão quando houver variação sensível entre os estratos definidos;
  \item[{2)}] Razões administrativas podem direcionar para a utilização de estratificação, visando
  principalmente caracterizar peculiaridades locais e regionais;
\end{enumerate}


%\newpage
\textbf{Exemplo de Amostragem Estratificada}
\vskip0.3cm

Castro et al (2016), realizou um estudo com informações retiradas de processos
judiciais de homens e mulheres adultos, que foram acusados de praticar agressão
sexual contra as crianças e adolescentes entre os anos de 2012 a 2014, período esse
escolhido por concentrar um maior número de processos no Sistema de Gestão de
Processos Judiciais - LIBRA, do Tribunal de Justiça do Estado do Pará. Nos quais,
foram escolhidos três municípios como população alvo, que supriam e respondiam aos objetivos do trabalho.\vskip0.3cm

\begin{table}[!htb]
    \centering
    {
    \caption{Número de Processos judiciais sobre práticas de agressão sexual com criancas e
adolescentes julgados pelo TJE em três municípios Paraenses no período de 2012 a 2014.}
    \label{amostras estratificada}
    \vspace{0.1cm}
\begin{tabular}{c|c|c}
  \hline\hline
  Municípios   & Nº de Processos &  Percentual \\
  \hline\hline
   Abaetetuba  & 47              & 6.69        \\
   Belém       & 555             & 82,22       \\
   Parauapebas & 73              & 10,81       \\
  \hline\hline
\end{tabular}}
\\
\hspace{-1.0cm}
\end{table}

De acordo com a tabela 2.3, vericou-se que, as maiorias dos processos estão
concentradas no município de Belém (82,2\%), 10,8\% em Parauapebas e apenas 6,9\%
em Abaetetuba. Porém, os processos não são digitalizados, contendo muitas páginas
cada um, e tendo que ser acessado em horários restritos no TJE, não podendo fazer
cópias dos mesmos. Dificultando, assim, o trabalho de catalogação de todos os 675
processos de agressão sexual, levando-se em conta, as caracteríssticas dos autores, das
vitimas e a tipificação das agressões. Com isso, devido a diversos fatores, optou-se
em trabalhar com uma amostra estatisticamente representativa da população em estudo.
\vskip0.3cm

Assim, para a um erro amostral de 5\%, tem-se a primeira estimativa para o tamanho da amostra em estudo dos processo judiciais.

\begin{equation}\label{nzero}
    n_{o}= \left[\frac{1}{\left(\epsilon_{0}\right)^2}\right] = \left[\frac{1}{\left(0,05
    \right)^2}\right]=400
\end{equation}

Em seguida, substituindo o resultado na formula 2.2,

\begin{equation}\label{nzero2}
    n=\left[\frac{N \times n_{0}}{N + n_{0}}\right]=
    \left[\frac{675 \times 400}{675 + 400}\right]=251
\end{equation}

Assim, faz-se necessário, apenas um amostra de 251 processos judiciais sobre
agressão sexual, sendo esta, estatiscamente representativa da população de 675.
\vskip0.3cm

Contudo, a população alvo está dividida em três municípios com quantidades
de processos diferentes em cada. Havendo a necessidade de levantar o tamanho da
amostra levando-se em conta que, o número de elementos sorteados em cada estrato
é proporcional ao numero de elementos no estrato, chamado Método de Amostragem
Estratificada Proporcional.
\vskip0.3cm

Assim, para se calcular o tamanho de amostra estratificada proprocional a cada
município (estrato) da população, utiliza-se a fórmula abaixo,


\newpage


%\begin{equation}
%\end{equation}

$$
n_{(h_{1)}}= n \times  \left[\frac{N_{(h_{1})}}{N} \right]= 251 \times \left[\frac{47}{675}\right]= 18
$$

$$ 	n_{(h_{2)}}= n \times \left[\frac{N_{(h_{2})}}{N} \right]=251 \times \left[\frac{555}{675}\right]= 206 $$

$$ 	n_{(h_{3)}}= n \times \left[\frac{N_{(h_{3})}}{N} \right]= 251 \times \left[\frac{73}{675}\right]= 27 $$

$$ n = \left[ n_{(h_{1})} + n_{(h_{2})} + n_{(h_{3})}\right]= 18+206+27 = 251
$$




\subsection{Amostragem Sistemática (AS)}


A amostragem sistemática consiste em sortear uma unidade da população e, a partir dela, para constituir
uma amostra de tamanho n, selecionar as unidades que ocupam de forma sequencial as suposições multiplas
de um determinado valor $k=\frac{N}{n}$ pré-estabelecido, onde N é o número de unidades total da população (QUEIROZ, 2012).\vskip0.3cm

No processo de amostragem sistemática a primeira unidade é escolhida ao acaso e as demais, a partir da inicial, selecionadas de modo sistemático a intervalos iguais e definidas, até atingir o tamanho da amostra desejada. O intervalo é estabelecido pela razão entre o tamanho da população e o tamanho da amostran (AYRES, 2010).\vskip0.3cm

De acordo com Loetsch et al (1973) a amostragem sistemática consiste em selecionar unidade de amostras a partir
de um esquema rígido e preestabelecido de sistematização segundo uma distribuição espacial equitativa ou mecânica, com os propósitos de cobrir a população em toda a sua extensão, e obter um modelo sistemático simples e uniforme.\vskip0.3cm


A amostragem sistemática apresenta uma diferença marcante com relação a amostragem aleatória simples, no que tange
ao número de amostras possíveis. Sendo recomendada para utilização quando os elementos da população encontram-se
ordenados (MARQUES, 2002).\vskip0.3cm


\newpage
Seja uma população composta por N unidades, e n represente o tamanho da amostra, sendo $N=kn$. Se a amostragem for sistematica, o número de amostras possíveis é igual a $k =\left[ \frac{N}{n} \right] $, enquanto que, em se tratando de uma amostra simples ao acaso sem reposição, o número de amostras possíveis (nap) será de acordo com a equação \ref{nap}. Então, teoricamente, uma amostra sistemática não pode ser analisada como se fosse uma amostra aleatória simples. \vskip0.3cm



De forma geral, na amostragem sistematica, utiliza-se a ordenação
natural dos elementos da população (prontuários, casas, ordem de
nascimento, etc), considerando N o tamanho da população e n o
tamanho da amostra, calcular o intervalo de amostragem, ou o pulo
sistematico, chamado k, por meio da formula $k=\frac{N}{n}$, senfo
k um numero inteiro mais proximo.\vskip0.3cm

Posteriormente, sorteia-se um numero entre um e k, chamado i,
sendo $0 < i \leq k$. Esse numero i será o primeiro elemento da
amostra. O segundo elemento da amostra será $i+k$, o terceiro
elemento será $i+2k$, e assim sucessivamente, de forma sistematica
até $i+(n-1)k$, ou seja, até completar o valor de n.

\vskip0.3cm


\textbf{Vantagens da Amostragem Sistemática}
\vskip0.3cm

A amostragem sistemática, quando comparada com a amostra simples, apresenta algumas vantagens, entre as quais:



\begin{enumerate}
  \item[{a)}] A seleção das unidades amostrais, na amostragem sistemática, é mais fácil e mais rápida, economizando tempo na obtenção dos dados de campo;
  \item[{b)}] A organização, a supervisão e a checagem (remedição) de algumas unidades de amostras se tornam operacioanalmente mais fáceis;
  \item[{c)}] O tamanho da população não precisa, necessariamente, ser conhecido;
  \item[{d)}] A redução de custo ocasionados pelo encaminhamento entre as unidades de amostras;
  \item[{e)}] As unidades se distribuem mais uniformemte na população, originalmente uma maior representatividade, tornando-se eficiente quando existe qualquer tendência ou concentração de certas características;
  \item[{f)}] È possível mapear a população;
  \item[{g)}] A amostragem sistemática pode ser associada  a amostragem por conglomerado;
\end{enumerate}

%\newpage


\textbf{Desvantagens da Amostragem Sistemática}
\vskip0.3cm


\begin{enumerate}
  \item[{a)}] A grande desvantagem da amostra sistematica ocorre
  quando a população apresenta dados com características \textbf{cíclicas}
  \textbf{ou periódicas}, pois corre-se o risco da amostragem refletir
  homogeneidade numa condição sabidamente heterogênea;
  \item[{b)}] Escolhe-se somente uma unidade de amostra simples, as demais não são independentes (estatisticamente, cada unidade não corresponde a um grau de liberdade). Assim, a variância da amostra e a da média não podem ser calculados através dos estimadores usuais;
  \item[{c)}] Escolhidos as amostras sistematicamente, todas as outras uniadades de amostra que não a integram têm probabilidade igual a zero de serem eleitas, enquanto as que integram-na possuem probabilidade 1 de seleção, ou seja, muitas unidades de amostras são, nesse caso, rejeitadas;
\end{enumerate}

\textbf{Exemplo de Amostragem Sistemática}
\vskip0.3cm

Retomando-se o exemplo de Castro et al (2016), ao qual calculou-se o tamanho da
amostra estratificada porporcional por municípios, ao fazer o levantamento deparou-se com elementos da população em forma de pilhas de processos judiciais ordenados
e sequenciais. Havendo assim, a necessidade de se aplicar o Método de Amostragem
Sistemático para se mapear a população de forma fácil e rápida.


\begin{table}[!htb]
    \centering
    {
    \caption{Número de Processos judiciais sobre práticas de agressão sexual com criancas e
adolescentes julgados pelo TJE em três municípios Paraenses no período de 2012 a 2014.}
    \label{amostras estratificada}
    \vspace{0.1cm}
\begin{tabular}{c|c|c}
  \hline\hline
  Municípios   & Nº de Processos &  Percentual \\
  \hline\hline
   Abaetetuba  & $N_{(h1)}=47$     & $n_{(h1)}= 6.69$        \\
   Belém       & $N_{(h2)}=555$    & $n_{(h2)}= 82,22$       \\
   Parauapebas & $N_{(h3)}=73$     & $n_{(h3)}= 10,81$       \\
   \hline\hline 
   Total       & $N_{(Total)}=675$ & $n_{(total)}=251$ \\ 
  \hline\hline
\end{tabular}}
\\
\hspace{-1.0cm}
\end{table}


%\vskip0.3cm

A amostragem sistemática consiste em selecionar unidade de amostras a partir
de um esquema rígido e preestabelecido de sistematização segundo uma distribuição
espacial equitativa ou mecânica, com os propósitos de cobrir a população em toda
a sua extensão, e obter um modelo sistemático simples e uniforme.
\vskip0.3cm

De forma geral, na amostragem sistemática, utiliza-se a ordenação natural dos
elementos da população (nesse caso os processos judiciais de agressores sexuais),
considerando N o tamanho da população (675) e n o tamanho da amostra (251),
calcular o intervalo de amostragem, ou o pulo sistemático, chamado k, por meio da
fórmula abaixo,

\begin{equation}
K= \left[ \frac{N}{n} \right]= \left[ \frac{675}{251} \right] = 3
\end{equation}

Posteriormente, sorteia-se um número entre 1 e k=3, chamado i, sendo que esse
número i serão o primeiro elemento da amostra. O segundo elemento da amostra será
(i+3), o terceiro elemento será (i+2)3, e assim sucessivamente, de forma sistemática
até i(n-1)3, ou seja, ate completar o valor de amostra n=251.\vskip0.3cm

Assim calcula-se o k, verifica-se que sorteando um número entre 1 e k=3, tem-se
o número i=2, assim, a amostra ficará da seguinte maneira:



\begin{table}[!htb]
    \centering
    {
    \caption{Amostragem sistemática sobre práticas de agressão sexual com crianças e adolescentes julgados pelo TJE em três municípios Paraenses no período de 2012 a 2014.}
    \label{amostras estratificada}
    \vspace{0.1cm}
\begin{tabular}{c|c|c|c|c|c|c|c|c|c}
  \hline\hline
   2  & 5  & 7  & 10 & 13 & 15 & 18 & 21 & 24 & 26 \\
   29 & 32 & 34 & 37 & 40 & 42 & 45 & 48 & 50 & 53 \\
   56 & 58 & 61 & 64 & 67 & ... & ... & ... & ... & 674 \\
  \hline\hline
\end{tabular}}
\\
\hspace{-1.0cm}
\end{table}




\newpage
\subsection{Amostragem Por Conglomerados ou Clusters}

A amostragem por conglomerados ou grupos é uma variação de qualquer processo de amostragem que, em vez de utilizar unidades de amostras individuais, usa um grupo ou conglomerado de pequenas subparcelas ou unidades secundárias (QUEIROZ, 2012).\vskip0.3cm




\begin{table}[!htb]
\centering
    {
    \caption{Esquema Geral sobre Amostragem por Conglomerados}
    \label{amostras estratificada}
    \vspace{0.1cm}
\begin{tabular}{c|c|c|c|c|c}
\hline\hline
\multirow{2}{*}{Subparcelas} & \multicolumn{5}{c}{Conglomerados}              \\
\cline{2-6}
                             & 1         & 2        &   3       & $\cdots$ & N            \\
\hline\hline
1                            & $y_{11}$  & $y_{21}$ &  $y_{31}$ & $\cdots$ & $y_{N_{1}}$  \\
2                            & $y_{12}$  & $y_{22}$ &  $y_{33}$ & $\cdots$ & $y_{N_{2}}$  \\
...                          & $\cdots$  & $\cdots$ &  $\cdots$ & $\cdots$ & $\cdots$     \\
M                            & $y_{1M}$  & $y_{2M}$ &  $y_{3M}$ & $\cdots$ & $y_{N_{M}}$   \\
\hline\hline
Totais                       &  $y_{1.}$ & $y_{2.}$ &  $y_{3.}$ &   $\cdots$       & $y_{N.}$   \\
\hline\hline
\end{tabular}}
\end{table}














\inic Apesar de a amostragem estratificada apresentar resultados
satisfatórios, a sua implementação é dificultada pela falta
de informações sobre a população para fazer a estratificação.\vskip0.3cm

Para poder contornar esse problema, pode-se trabalhar com o
esquema de levantamento chamado amostragem por conglomerados.\vskip0.3cm

Os conglomerados são definidos em razão da experiência
do gestor ou do pesquisador. Geralmente, podemos definir os
conglomerados por fatores geográficos, como bairros e quarteirões.
A utilização da amostragem por conglomerados possibilita uma
redução significativa do custo no processo de amostragem.\vskip0.3cm

A amostragem por conglomerados ou também chamada de clusters, é usada quando a população
pode ser dividida em subpopulações heterogêneas.\vskip0.3cm

Uma amostragem por conglomerado é indicada quando:


\begin{itemize}
  \item Não se possui uma lista contendo todos os nomes dos elementos da população;
  \item Existe grande heterogeneidade entre os elementos da população-alvo;
  \item É preciso fazer entrevistas ou observações em grandes áreas geográficas;
\end{itemize}


As unidades de amostragem ou grupos podem ser espaçadas como ocorre naturalmente em unidades geográficas ou físicas (por exemplo, estados, delegações ou distritos); em base  a uma organização como escolas, grau escolar; ou serviço telefônico, tais como códigos de área ou alterar o código de área do serviço dos números de telefone.\vskip0.3cm

Duas grandes dimensões são utilizadas para classificar os diferentes tipos de amostragem por conglomerados. Uma baseia-se no número de fases do projeto da amostra, e o outra na representação proporcional de grupos na amostra total..\vskip0.3cm


\section{Amostragem Não Probabilística}


Os métodos de amostragem não probabilística, dirigido ou também chamados de não aleatória, são métodos ad-hoc de caracter pragmático ou intuitivo e são largamente utilizados, pois possibilitam um
estudo mais rápido e com menores custos.




\subsection{Amostragem por Conveniência ou Acidental}


\inic A amostragem por conveniência é adequada e freqüentemente utilizada para geração de idéias em pesquisas exploratórias, servindo como base para geração de hipóteses e insights.\vskip0.3cm



A amostra por conveniência é empregada quando se deseja obter informações de maneira rápida e barata. Segundo diversos autores, uma vez que esse procedimento consiste em simplesmente contatar unidades convenientes da amostragem, é possível recrutar respondentes tais como estudantes em sala de aula, mulheres no shopping, alguns amigos e vizinhos, entre outros. Este método também pode ser empregado em pré-testes de questionários.\vskip0.3cm


As amostras por conveniência são o tipo de amostragem menos confiável pois o pesquisador seleciona a amostra conforme sua conveniência, havendo pouco rigor na seleção, onde não há como saber se todas as pessoas incluídas na amostra são representativas da população.\vskip0.3cm



Nos casos de amostragem por conveniência, a diferença entre os valores da população de interesse e os valores da amostra é desconhecida, em termos de tamanho e de direção. Não é possível mensurar os erros desta amostragem e não é possível fazer nenhuma declaração definitiva ou conclusiva sobre os resultados obtidos, não sendo recomendadas para estudos causais e descritivos.




\subsection{Amostragem Intencional ou por Julgamento}

A seleção de amostras intencionais ou por julgamento são realizadas de acordo com o julgamento do pesquisador, sendo comum a escolha de experts (profissionais especializados) usados para escolher emementos "típicos" e "representativos" para uma amostra.
\vskip0.3cm


A abordagem da amostragem por julgamento pode ser útil quando é necessário incluir um pequeno número de unidades na amostra. O método de julgamento é muito utilizado para a escolha de uma localidade "representativa" de um país na qual serão realizadas outras pesquisas, sendo algumas vezes até preferida em relação à seleção de uma localidade por métodos aleatórios.

%\newpage 
\subsection{Amostragem por Quotas ou Proporcional}


A amostra por quotas constitui um tipo especial de amostra intencional, em que o pesquisador procura obter uma amostra que seja similar à população sob algum aspecto.\vskip0.3cm

A seleção de amostra por quotas é a forma mais usual de amostragem não probabilística. Neste caso, são consideradas várias características da população como gênero, idade, raça e renda familiar, incluindo proporções similares de pessoas com as mesmas características.\vskip0.3cm


A idéia de amostragem por quotas sugere que se as pessoas são representativas em termos de características, elas também poderão ser representativas em termos da informação procurada pela pesquisa. Depois de serem identificadas as proporções de cada tipo a ser incluído na amostra, o pesquisador estabelece um número ou quota de pessoas que possuem as características determinadas e que serão contatadas pela pesquisa. O entrevistador recebe instruções para continuar a amostragem até que a quota necessária tenha sido atingida em cada estrato.\vskip0.3cm


Uma pesquisa com amostragem por quotas poderá ser utilizado e trazer bons resultados quando as características relevantes para controle e delineamento da amostra forem conhecidas, estiverem disponíveis ao pesquisador, estiverem relacionadas ao objeto de estudo e se constituírem em poucas categorias.
