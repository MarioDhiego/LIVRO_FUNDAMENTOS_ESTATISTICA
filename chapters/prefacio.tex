\chapter*{Prefácio}

\inic A idéia de escrever um texto introdutório sobre fundamentos de estatística para gestão pública surgiu da necessidade de se divulgar o potencial dessa teoria tanto no
seu aspecto estatístico-matemático quanto na sua aplicação e
interpretação em diversas áreas do setor público.\vst

A importância da estatística para o gestor público pode ser vista através da sua utilização ao nível do Estado, de Organizações Sociais e Profissionais, do cidadão comum e ao nível acadêmico. Não restam dúvidas de que uma base qualificada de informações é fundamental para a adequada gestão das políticas públicas.\vst  

A estatística fornece ferramentas importantes para que os governos possam definir melhor suas metas, avaliar sua performance, identificar seus pontos fortes e fracos e atuar na melhoria contínua das políticas públicas.
\vst

Nossa maior preocupação foi a de escrever um texto que pudesse ser
utilizado não só pelos estatísticos, mas também por profissionais de outras áreas. O sucesso da Estatística passa necessariamente
pelo trabalho conjunto de especialistas dessas várias áreas. \vst

Muito do material e ideias apresentadas nesse livro foram
desenvolvidos durante o planejamento e a análise de diversas experiências técnicas e acadêmicas, ao longo de 20 anos trabalhando na Gestão Pública Estadual.
\vst 


Devido a enorme abrangência da Estatística, procuramos detalhar os pontos mais interessantes para um texto introdutório e fornecer o maior número possível de referências bibliográficas que cobrissem os outros pontos.\vst

O profissional que domina os princípios estatísticos tem em suas mãos uma poderosa ferramenta que poderá ser sua aliada ao longo da carreira. As aplicações são diversas, e ter a compreensão desse contexto é a primeira grande lição.
\vst

\newpage
Este livro tem finalidade didática, sem a preocupação com o aprofundamento dos assuntos, o que provavelmente afastaria os estudantes iniciantes leigos no assunto. Para tanto, foi escrito utilizando linguagem de programação \textit{open source} chamada \textbf{\LaTeX} (é um conjunto de comandos adicionais (macros) para o código \TeX), elaborado em meados da década de 80 por \textbf{Leslie Lamport}. Em sua modernidade, utilizou-se um "Ambiente de Desenvolvimento Integrado(IDE)" para manipulação dos scripts, chamado \textbf{OvearLeaf}. \vst


Para facilitar a análise dos dados e a construção dos gráficos foram introduzidos vários exemplos elaborados com os recursos computacionais do Software Estatístico \textit{open source} \textbf{$R_{4.3.1}$} versão 64 bit for Windows. 
\vst





%Este trabalho foi parcialmente financiado pelo DETRAN-PA.

\vst
\vst
\vst

\begin{centering}

\vst

%\textbf{Semana Nacional de Trânsito (Setembro 2023)} 
\vsm

Mário Diego Rocha Valente (Analista de Trânsito/Detran-PA) \\
Geovana Raio Pires (Técnica em Gestão Pública/Seplad-PA)\\
Héliton Ribeiro Tavares (Prof Dr. Titular/UFPA)\\
Waldenei Travassos De Queiroz (Prof Dr. Titular/UFRA)\\



\end{centering}
