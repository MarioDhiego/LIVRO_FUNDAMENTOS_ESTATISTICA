\chapter*{Prefácio}

\inic A idéia de escrever um texto introdutório sobre fundamentos de estatística para gestão pública surgiu da necessidade de se divulgar o potencial dessa teoria tanto no
seu aspecto estatístico-matemático quanto na sua aplicação e
interpretação em diversas áreas do setor público.\vskip0.3cm



A importância da estatística para o gestor público pode ser vista através da sua
utilização ao nível do Estado, de organizações sociais e profissionais, do cidadão comum e ao nível acadêmico. Não restam dúvidas de que uma base de informações qualificada é fundamental para a adequada gestão das políticas públicas.  A estatística fornece ferramentas importantes para que os governos possam definir melhor suas metas, avaliar sua performance, identificar seus pontos fortes e fracos e atuar na melhoria contínua das políticas públicas.
\vskip0.3cm

Nossa maior preocupação foi a de escrever um texto que pudesse ser
utilizado não só pelos estatísticos, mas também por profissionais de outras áreas. O sucesso da Estatística passa necessariamente
pelo trabalho conjunto de especialistas dessas várias áreas. Devido
a enorme abrangência da Estatística. Nesse sentido, procuramos detalhar
alguns pontos que achamos importantes. \vst

Muito do material e idéias apresentadas nesse livro foram
desenvolvidos durante o planejamento e a análie de diversas experiências técnicas e acadêmicas, ao longo de 15 anos trabalhando na Gestão Pública Estadual.
\vst 

\newpage


Devido a enorme abrangência da Estatística, procura-se detalhar os pontos
que achamos mais interessantes para um texto introdutório e
fornecer o maior número possível de referências bibliográficas que
cobrissem os outros pontos.\vst

Este trabalho foi parcialmente financiado pelo DETRAN-PA.

\vst

\begin{centering}

\vst

Dezembro 2023 
\vsm

Mário Diego Rocha Valente \\
Giovana Raio Pires \\
Heliton Ribeiro Tavares \\
Waldenei Travassos de Queiroz \\



\end{centering}
